%
%   AEP 4380 Final Project
%   Fall 2014
% 
%   12-dec-2014 Austin Liu
%   ayl42
%
\documentclass[11pt,letterpaper]{article}

\setlength{\textwidth}{6.5in}
\setlength{\textheight}{9in}
\setlength{\topmargin}{-0.5in}
\setlength{\oddsidemargin}{0in}
\setlength{\evensidemargin}{0in}

\setlength{\parskip}{0.1in}
\setlength{\parindent}{0.5in}

\pagestyle{empty}  % get rid of page numbers
\pagestyle{myheadings}
\markboth{{ AEP4380}, Final Project, Fall 2014}
{{ AEP4380}, Final Project, Fall 2014}

%\usepackage[dvipdfm]{graphics}  % for complicated illustrations
\usepackage[pdftex]{graphicx}  % for complicated illustrations with non-eps graphics
%\usepackage[dvips]{graphics}  % for complicated illustrations
\usepackage{subcaption}
\usepackage{amsmath}
\usepackage{url}
\usepackage{braket}
\usepackage[backend=bibtex]{biblatex}
\bibliography{writeup.bib}

\begin{document}
\title{Hartree-Fock Calculations}
\author{Yu Austin Liu}
\date{\today}
\maketitle


\section{Introduction and Theoretical Background}
The Hartree-Fock method is a well known method for calculating energy levels
in many-electron atoms. For the final project, I first calculated
the energy levels of helium.

For helium, the Hamiltonian can be written as :
\begin{align}
\hat{H} = \hat{H}_1 + \hat{H}_2 + \frac{e^2}{r_{12}}
\end{align}

where 

\begin{align}
\hat{H}_i = -\frac{\hbar^2}{2m_e}\nabla_i^2 - \frac{Ze^2}{r_i}
\label{hamiltonian}
\end{align}

The del operator is
$\nabla_i$ = ($\frac{\partial}{\partial x_i}$, $\frac{\partial}{\partial y_i}$,
$\frac{\partial}{\partial z_i}$). Note that the nucleus of the
atom has already been placed at the origin and ``frozen'' to simplify things
(the Born-Oppenheimer approximation). The $\frac{e^2}{r_{12}}$ term, where
$r_{12} = |\mathbf{r}_1 - \mathbf{r}_2|$ is particularly problematic.

\section{Hartree-Fock method}

To make the problem tractable, assume that the spatial part of the wavefunction
can be made separable. Since electrons are fermions, 
we need to include a separate spin coordinate,
$\sigma$, in addition to the spatial coordinates $\mathbf{r}$,
, so let $\mathbf{x} = (\mathbf{r}, \sigma)$


\begin{align}
\Psi(\mathbf{x}_1,\mathbf{x}_2) = \psi_1(\mathbf{x}_1)\psi_2(\mathbf{x}_2)
\end{align}

However, by the Pauli exclusion principle, the wavefunction must be
antisymmetric, i.e. we have to take a linear combination of both
Hartree products by choosing:

\begin{align}
\Psi(\mathbf{x}_1,\mathbf{x}_2) = 
\frac{1}{\sqrt{2}}[\psi_1(\mathbf{x}_1)\psi_2(\mathbf{x}_2) - 
                   \psi_1(\mathbf{x}_2)\psi_2(\mathbf{x}_1)] = 
\frac{1}{\sqrt{2}}
\begin{vmatrix} 
\psi_1(\mathbf{x}_1) & \psi_2(\mathbf{x}_1) \\ 
\psi_1(\mathbf{x}_2) & \psi_2(\mathbf{x}_2) 
\end{vmatrix}
\label{slaterdet}
\end{align}

Following the notation in \cite[p.~152]{griffiths05}, \cite{david06} and
\cite[p.~68]{koonin90}, closely, we take

\begin{align}
\psi_1(\mathbf{x}) = 
\frac{1}{\sqrt{4\pi}}
\frac{u(r)}{r}
\alpha(\sigma)
\end{align}

and

\begin{align}
\psi_2(\mathbf{x}) = 
\frac{1}{\sqrt{4\pi}}
\frac{u(r)}{r}
\beta(\sigma)
\end{align}


where $\alpha(\sigma) = \Braket{\sigma | \alpha}$ gives the spin component 
of the wavefunction $\psi_1$ and
$\beta(\sigma)$ gives the spin component of the wavefunction for
$\psi_2$. The wavefunction is assumed to have separable
azimuthal and polar angular dependence, i.e. the dependence on the coordinates
$\phi$ and $\theta$ can be separated out as in the case of hydrogen, this
is detailed in, for instance, \cite{griffiths05}.



From Equation \eqref{slaterdet} we find that
\begin{align}
\frac{u(r_1)u(r_2)}{4\pi r_1r_2\sqrt{2}}
\left[\alpha(\sigma_1)\beta(\sigma_2)-\alpha(\sigma_2)\beta(\sigma_1)\right]
\end{align}

Since the Hamiltonian does not involve spin (in non-relativistic
quantum mechanics), we use the Schr\"{o}dinger equation,
$\hat{H}\Psi = E\Psi$
and now write:

\begin{align}
\left(\hat{H}_1 + \hat{H}_2 + \frac{e^2}{r_{12}}\right)
\frac{u(r_1)u(r_2)}{4\pi r_1r_2} =
E \frac{u(r_1)u(r_2)}{4\pi r_1r_2} 
\end{align}


Taking an inner product over the 1 coordinate, we obtain


\begin{align}
\int_0^\infty 4\pi r_1^2 
\frac{u^*(r_1)}{r_1}
\left(\hat{H}_1 + \hat{H}_2 + \frac{e^2}{r_{12}}\right)
\frac{u(r_1)u(r_2)}{4\pi r_1r_2}
dr_1
=
E \int_0^\infty 4\pi r_1^2 
\frac{u^*(r_1)}{r_1}
\frac{u(r_1)u(r_2)}{4\pi r_1r_2}
dr_1
\end{align}

Let

\begin{align}
\Braket{1|\hat{H}_1|1} = 
\int_0^\infty 
u^*(r_1) \hat{H}_1\left(u(r_1)\right)
dr_1
\end{align}

Hence obtain:

\begin{align}
\Braket{1|\hat{H}_1|1}\frac{u(r_2)}{r_2} + 
\Braket{1|1}\hat{H}_2 \left(\frac{u(r_2)}{r_2}\right)+
\frac{u(r_2)}{r_2}\int_0^\infty u^*(r_1)
\left(\frac{e^2}{r_{12}}\right)u(r_1)dr_1
=
E \Braket{1|1} \left(\frac{u(r_2)}{r_2}\right)
\end{align}

Let 
\begin{align}
\int_0^\infty u^*(r_1)
 \left(\frac{e^2}{r_{12}}\right)u(r_1)dr_1
=
\Braket{1|V|1}
\end{align}


Now expand the Hamiltonian and formulating the equation in terms of a
Schr\"{o}dinger-like equation \cite[p.~69]{koonin90},
we obtain:

\begin{align}
\left(\hat{H}_2 + \frac{1}{2}\Braket{1|V|1} \right)\frac{u(r_2)}{r_2} = 
\left(E - \Braket{1|\hat{H}_1|1} - \frac{1}{2}\Braket{1|V|1}\right)
\frac{u(r_2)}{r_2}
\end{align}

Let
\begin{align}
\varepsilon_2 = \left(E - \Braket{1|\hat{H}_1|1} - 
\frac{1}{2}\Braket{1|V|1}\right)
\end{align}

Note that this choice of $V$ essentially ``splits'' the
electron-electron repulsion term into two, so that
the raising of the ground state energies due to the inter-electron
repulsion term is split evenly between the two electrons and
both $\varepsilon_1$ and $\varepsilon_2$ give the ground state energy of a
helium electron in the ground state.

Using Equation \eqref{hamiltonian}, and the formula for the Laplacian
in spherical coordinates, we obtain:
\begin{align}
-\frac{\hbar^2}{2mr_2}\frac{d^2}{dr_2^2} u(r_2)
+\left[
- \frac{Ze^2}{r_2}
 + \frac{1}{2}\int_0^\infty u^*(r_1) \left(\frac{e^2}{r_{12}}\right)u(r_1)dr_1
\right]
\frac{u(r_2)}{r_2}
=
\varepsilon_2 \frac{u(r_2)}{r_2}
\end{align}

Interchanging indices gives a similar eigenvalue equation for $u(r_1)$.
Simplifying:
\begin{align}
\left[
-\frac{\hbar^2}{2m}\frac{d^2}{dr_2^2}
- \frac{Ze^2}{r_2}
 + \frac{1}{2}\int_0^\infty u^*(r_1) \left(\frac{e^2}{r_{12}}\right)u(r_1)dr_1
\right]
u(r_2)
=
\varepsilon_2 u(r_2)
\label{ownradialse}
\end{align}

Assuming spherical symmetry, 
let the charge density of electron 1 due to its wavefunction be
\begin{align}
\rho(r_1)= \frac{u^*(r_1)u(r_1)}{4\pi r_1^2}
\end{align}

Then the potential due to this electron (note that $r_{12} = |r_2 - r_1|$,
and factors of $4\pi r_1^2$ cancel)
is given by:
\begin{align}
V(r_2) = e^2\int_0^\infty \frac{\rho(r_1)}{|r_2-r_1|}dr_1
\end{align}

The potential is solved using Poisson's equation \cite[p.~49]{koonin90}.
The 2 subscript is dropped for clarity.
\begin{align}
\frac{1}{r^2}\frac{d}{dr}\left(r^2\frac{d}{dr}V(r)\right)
= -4\pi e^2\rho(r)
\end{align}

The substitution $V(r) = v(r)/r$ reduces this to:
\begin{align}
\frac{d^2}{dr^2}v(r)
= -4\pi e^2r\rho(r)
\end{align}

which can be solved via Numerov's method (outlined below).

Equation \eqref{ownradialse} 
is of a form similar to the form of the radial Schr\"{o}dinger
equation, Equation \eqref{radialse}, 
obtained after separation of variables on the
Schr\"{o}dinger equation in spherical coordinates
\cite[p.~140]{griffiths05}.
\begin{align}
 \left(
  -\frac{\hbar^2}{2m}\frac{d^2}{dr^2} 
  +\left[V + \frac{\hbar^2}{2m}\frac{l(l+1)}{r^2} \right]
 \right)u(r) = Eu(r)
\label{radialse}
\end{align}
Note that for helium, $l=0$ for both of the electrons since both are in the
$n=0$ shell \cite[p.~161,164]{griffiths05}.


\section{Numerov method}
The Numerov method is a numerical method for solving 
equations of the form:
\begin{align}
u''(r) + w(r)u(r) = s(r)
\label{ldeq2}
\end{align}
For example, equation \eqref{radialse} is of this
form with
\begin{align}
w(r) = \frac{2m}{\hbar^2}\left[E - V(r) \right]
\end{align}
(where this $V$ is not to be confused with the one previously introduced).

Let $h$ be the step size and 
replace $u''(r)$ by a finite difference approximation \cite[p.~50]{koonin90}
\begin{align}
u''_n =
\frac{u_{n+1} - 2u_{n} + u_{n-1}}{h^2}
- \frac{h^2}{12}u^{(4)}_n + \mathcal{O}(h^4)
\end{align}

So Equation \eqref{ldeq2} becomes:
\begin{align}
\frac{u_{n+1} - 2u_{n} + u_{n-1}}{h^2}
- \frac{h^2}{12}u^{(4)}_n + \mathcal{O}(h^4) + (wu)_n = s_n
\label{fdur}
\end{align}

To obtain the $u^{(4)}(r)$ term, we differentiate Equation \eqref{ldeq2}
twice
\begin{align}
\frac{d^2}{dr^2}\left[u''(r) + w(r)u(r) - s(r) \right] = 0 
\end{align}
so
\begin{align}
u^{(4)}_n =  -\frac{(wu)_{n+1} - 2(wu)_{n} + (wu)_{n-1}}{h^2}
            +\frac{s_{n+1}-2s_n+s_{n-1}}{h^2}
            + \mathcal{O}(h^2)
\label{fdwur}
\end{align}

Inserting Equation \eqref{fdwur} into Equation \eqref{fdur}, we
can manipulate and obtain:
\begin{multline}
\left(1+ \frac{h^2}{12}w_{n+1}\right)u_{n+1} 
- 2\left(1- \frac{5h^2}{12}w_n\right)u_n
+ \left(1+ \frac{h^2}{12}w_{n-1}\right)u_{n-1}  
= \frac{h^2}{12}\left(s_{n+1}+10s_n+s_{n-1}\right) + \mathcal{O}(h^6)
\end{multline}

This can be further manipulated to get a recursion relation forward or backward
for $u_{n+1}$ or $u_{n-1}$.
The Numerov algorithm is first tested on the hydrogen atom to trace
the 1s wavefunction in htest.cpp, the first program in the source
code listing. The forward Numerov
method is used for the first half of the wavefunction starting
from $r = 0$ and
the backward Numerov method is used for the second half starting from
$r = 3.0$, the cutoff radius.
The plot is shown in Figure \ref{Rofrh}.
\begin{figure}[htb!]
\begin{center}
\includegraphics[height=3.2in]{../img/Rofr_h.png}
\end{center}
\caption{Plot of numerical and analytical radial hydrogen wavefunction 
(after normalization with $Y_0^0(\theta,\phi) = \frac{1}{\sqrt{4\pi}}$ term
\label{Rofrh}}
\end{figure}

The energy calculation subroutine is also tested on the 1s
hydrogen wavefunction, giving -13.605 eV for the total energy and
a ratio of 0.4997 for the kinetic to potential energy ratio. We expect a ratio
of 0.5 by the virial theorem though there may be roundoff error that
has accumulated due to the integration using Simpson's rule.

\section{Self-consistency calculations}

One variational treatment uses a trial wavefunction which is 
of the form of that of the
hydrogen atom, parameterized by the effective charge, $Z^*$.

\begin{align}
u(r) = 2 \left(\frac{Z^*}{a}\right)^{1/2} \frac{Z^*r}{a} e^{-Z^*r/a}
\label{1swav}
\end{align}

$a$ above is the Bohr radius.


The next step is to use the variational principle, i.e. the expectation
value of the energy of any wavefunction will be greater than or equal to the
expectation value of the energy of the actual wavefunction, to refine guesses
to the wavefunction.

The iteration scheme outlined by Koonin in \cite[p.~74]{koonin90} 
can be summarized as follows:

\begin{enumerate}
\item Guess an initial wavefunction
\item Use the normalization condition for the wavefunction 
to solve for the potential generated by the initial wavefunction and 
calculate the total energy of the sustem.
\item Find a new wavefunction and its eigenvalue by solving the 
Schr\"{o}dinger equation with the
new potential and normalizing it again.
\item Calculate the new potential and new total energy, then repeat 
3. and 4. until the total energy has converged within the required tolerance.
\end{enumerate}

\section{Results and Conclusion}

The procedure outlined above  
is done in hf1d\_he.cpp, the second program 
in the source code listing. First a trial wavefunction of the form in 
Equation \eqref{1swav} is chosen with $Z$ = 2. Testing with
different values of $Z$ still gives the same converged result. 
Next, the self-consistency loop is run until the energy converges to a minimum
value and the old and new energies are within 0.001 eV of each other. 
These energies are tabulated in Table \ref{etable}. Note that the energy
given is the energy per electron and must be multiplied by two to get the
ground state energy of helium.
\begin{table}[h!]
\begin{tabular}[h!]{| l | c | c | r | }
  \hline
  Stepsize & Cutoff radius & Energy (eV) & $|E_k/E_p|$ \\
  \hline
  0.004 & 4.0 & -40.14  & 0.556 \\
  0.002 & 4.0 & -39.28  & 0.55 \\
  0.001 & 4.0 & -38.887 & 0.548 \\
  0.0005 & 4.0 & -38.7 & 0.547 \\
  \hline
\end{tabular}
\caption{\label{etable}
Table of results for calculations}
\end{table}

Plots of $u(r)$ and $R(r)$ were also made in Figures \ref{uofrhe} and
\ref{Rofrhe} respectively. Konnin lists the calculated energy to be 
-78.88 eV so the value obtained with the smallest stepsize, -38.7(2) = -77.4 eV
is about 2\% off from this value. This value is also roughly the value obtained
using the variational principle \cite[p.~314-315]{griffiths05}. 
While $R(r)$ does not seem to have 
changed much at the end of the iterative cycle, since $u(r)/r = R(r)$ and
$u(r)$ has become ``flatter'' at the end of the iterative cycle, the
inter-electron repulsion effectively results in the electrons being further
away from the nucleus, which is expected. It is somewhat dissapointing that
the $|E_k/E_p|$ ratio remains somewhat off, which could indicate some roundoff
error in the algorithm or something unaccounted for.

\begin{figure}[h!]
\begin{center}
\includegraphics[height=3.2in]{../img/uofr_he.png}
\end{center}
\caption{Plot of initial and final $u(r)$ for He
\label{uofrhe}}
\end{figure}

\begin{figure}[h!]
\begin{center}
\includegraphics[height=3.2in]{../img/Rofr_he.png}
\end{center}
\caption{Plot of initial and final $R(r)$ for He
\label{Rofrhe}}
\end{figure}


\pagebreak
\pagebreak
\pagebreak
\printbibliography[heading=none]

\appendix
\section{Source Code Listing}
\begin{verbatim}
/* AEP 4380 Final Project

   Test of Numerov's method and
   various methods to calculate kinetic and
   potential energy for a hydrogen 1s wavefunction.

   Austin Liu
   ayl42
   2014-12-15

   run on a core i3 with g++ 4.7.2 on the Debian 7.6 Linux distribution.
*/

#include <cstdlib>
#include <cmath>

#include <iostream>
#include <fstream>
#include <iomanip>

using namespace std;

const double pi= 3.14159265358979;
const int N = 10001;
const double a = 0.5299; // Bohr radius in Angstroms a = hbar^2/(me^2)
const double esq = 14.409; // e^2 in eV-Angstroms
const double hbsq_m = 7.6359; // hbar^2/m in eV-Angstroms^2
const double ri = 0.0;
const double rf = 3.0;
const double h = (rf-ri)/(N-1);
const double Z = 1.0;
int i;
double Ep, Ek, E;
double r [N];
double u [N];
double usq [N];
double dudr [N];
double dudrsq [N];
double w [N];
double R [N];
double Rnum [N];
double eptmp [N];
  
/* The hydrogenic 1s wavefunction */
double f(double r){
  double t = (Z/a);
  return sqrt(t/pi)*t*exp(-t*r);
}

/* Weighting function to solve for the
   wavefunction numerically, i.e. w(r) in the 
   expression for Numerov's method. */
double g(double r){
  return (2/hbsq_m)*((Z*esq/r) - 13.605);   
}

/* Sets the initial conditions*/
void set_ics(double (&u)[N]){
  u[0]= 0.0;
  u[1]= 2*sqrt(Z/a)*(Z/a)*h*exp(-Z*h/a) ; 
  // try exact values
  u[N-1]= 2*sqrt(Z/a)*(Z/a)*h*(N-1)*exp(-Z*h*(N-1)/a);
  u[N-2]= 2*sqrt(Z/a)*(Z/a)*h*(N-2)*exp(-Z*h*(N-2)/a);
}

/* Update usq after determining a new u(r) */
void updateusq(){
  for(i=0; i<N; i++){
    usq[i]= u[i]*u[i];
  }  
}

/* Uses the Numerov method to update the (k+1)th term 
   k must be in range (0, N-1). Assumes terms
   u[0] through u[k] are already initialized and
   w is initialized. */
void numerovf(int k, double (&u) [N], double w [N]){
  double a = h*h/12.0;
  u[k+1]= ( 2*u[k]-u[k-1]-a*(10*w[k]*u[k] + w[k-1]*u[k-1]) )/(1+a*w[k+1]);
}

/* Uses the Numerov method to update the (k-1)th term 
   k must be in range (0, N-1). Assumes terms
   u[N-1] through u[k] are already initialized and
   w is initialized. */
void numerovb(int k, double (&u) [N], double w [N]){
  double a = h*h/12.0;
  u[k-1]= ( 2*u[k]-u[k+1]-a*(10*w[k]*u[k] + w[k+1]*u[k+1]) )/(1+a*w[k-1]);
}

/* Initialize values in r and w */
void initialize(double (&r) [N], double (&w) [N]){
  for(i=0; i<N; i++){
    r[i]= h*i;
    w[i]= g(r[i]);
    R[i]= f(r[i]);
  }  
  // use linear extrapolation because w[0] goes to infinity
  // due to 1/r term in denominator.
  w[0]= 2*w[1]-w[2]; 
}

/* Compute integral using Simpson's rule, N must be odd */
double simpint(double f [N]){
  double sum;
  sum= f[0];
  for(i=1; i<N-1; i+=2){
    sum+= 4*f[i];
  }
  for(i=2; i<N-2; i+=2){
    sum+= 2*f[i];
  }
  sum += f[N-1];
  sum *= h/3;
  return sum;
}

/* Updates the derivative of a function using numerical differentiation
   Assumes f is a function of r. */
void updatederiv(double f [N], double (&dfdr) [N]){  
  // have to use forward difference
  dfdr[0]= (f[1]-f[0])/h;
  // use central difference for middle ones
  for (i=1; i< N-1; i++){
    dfdr[i]= 0.5*(f[i+1]-f[i-1])/h;
  }
  // use backward difference
  dfdr[N-1]= (f[N-2]-f[N-1])/h;
}

/* Compute the kinetic energy of a wavefunction */
double calcEk(double u [N]){
  double Ek;
  updatederiv(u, dudr);
  for (i=1; i<N; i++){
    dudrsq[i]= dudr[i]*dudr[i];  
  }
  Ek = hbsq_m*0.5*simpint(dudrsq);
  return Ek;
}

/* Compute the potential energy of a wavefunction using a 
   predetermined potential*/
double calcEp(double u [N]){
  for (i=1; i<N; i++){
    eptmp[i]= (-Z*esq/r[i])*usq[i];
  }
  return simpint(eptmp);
}


void writetofile(){
  ofstream fp;
  fp.open("hf1d_h.dat");
  if (fp.fail()){
    cout << "cannot open file" << endl;
  }
  for (i=0; i<N; i++){
    fp << r[i] << setw(15) << R[i] << setw(15) 
       << Rnum[i] << endl;
  }
}

/* Solve for the hydrogenic wavefunction numerically */
void calcu(){
  // trace forward for first half
  for (i= 1; i<N/2; i++){
    numerovf(i, u, w);
  }
  // trace backward for second half
  for (i= N-2; i>N/2; i--){
    numerovb(i, u, w);
  }
  // normalization required
  for (i=0; i<N; i++){
    usq[i]= u[i]*u[i];
  }
  double m = sqrt(simpint(usq));
  cout << "norm. const. = " << m << endl;
  for (i=0; i<N; i++){
    u[i]/= m;
    usq[i]= u[i]*u[i];
  }
  cout << "normalized to: " << simpint(usq) << endl;
}

int main(){

  initialize(r,w);
  set_ics(u);

  // for (i= 1; i<N-1; i++){
  //   numerovf(i, u, w);
  // }
  calcu();
  updateusq();
  Ep= calcEp(u);
  Ek= calcEk(u);
  E= Ep+Ek;
  cout << "kinetic energy is: " << Ek << endl;
  cout << "potential energy is: " << Ep << endl;
  cout << "total energy is: " << E << endl;

  // generate Rnum values for comparison
  for (i= 1; i<N; i++){
    Rnum[i] = u[i]/(r[i]*sqrt(4*pi));
  }  
  Rnum[0] = 2*Rnum[1]-Rnum[2];

  writetofile();
  return (EXIT_SUCCESS);
} // end main()
\end{verbatim}
\vspace{1.0in}
\begin{verbatim}
/* AEP 4380 Final Project

   Calculation of the ground state energy of helium
   using a self-consistent loop.

   Austin Liu
   ayl42
   2014-12-15

   run on a core i3 with g++ 4.7.2 on the 
   Debian 7.6 Linux distribution.
*/


#include <cstdlib>
#include <cmath>

#include <iostream>
#include <fstream>
#include <sstream>
#include <iomanip>

using namespace std;

const double pi = 3.14159265358979;
const double a = 0.5299; // Bohr radius in Angstroms a = hbar^2/(me^2)
const double hbsq_m = 7.6359; // hbar^2/m in eV-Angstroms^2
const double esq = 14.409; // e^2 in eV-Angstroms
const double Z = 2.00;
const int N = 2001;
const double ri = 0.0;
const double rf = 4.0; // in Angstroms
const double h = (rf-ri)/(N-1);
int i;
int numiter; // counter
double r [N];
double R [N]; // for R(r)
double u [N]; // for u(r) = r*R(r)
double usq [N];
double unew [N];
double unewsq [N];
double w [N]; // to be used as a weighing function for Numerov's method
double dudr [N];
double dudrsq [N];
double eptmp [N]; // for holding the potential energy intergrand
double V [N]; // for the potential energy V(r)
double v [N]; // v(r) = rV(r) - easier to solve for this first.
// double rho [N]; // for the charge density distribution
double f [N]; // for the function on the right hand side of
              // Poisson's equation,  i.e. -4*pi*r*(rho(r))
double blank [N]; // to handle optional arguments into Numerov
double Ek;
double Ep;
double E;
double Eold;
  
/* A trial radial part of the wavefunction
   If R(r) = u(r)/((sqrt(4pi)r) then this is u(r) */
double uradial(double r){
  double t = (Z/a);
  return 2*sqrt(t)*t*r*exp(-t*r);  
}

/* A trial radial part of the wavefunction
   If R(r) = u(r)/((sqrt(4pi)r) then this is R(r) */
double Rradial(double r){
  double t = (Z/a);
  return sqrt(t/pi)*t*exp(-t*r);  
}

/* Initialize values in r and w */
void initialize(){
  for(i=0; i<N; i++){
    r[i]= h*i;
    u[i]= uradial(r[i]);
    R[i]= Rradial(r[i]);
    usq[i]= u[i]*u[i];
    blank[i]= 0;
  }  
}

/* Update usq after determining a new u(r) */
void updateusq(){
  for(i=0; i<N; i++){
    usq[i]= u[i]*u[i];
  }  
}

/* Computes integral using the trapezoid rule */
double trapint(double f [N]){
  double sum;
  sum= f[0]/2;
  for(i=1; i<N-1; i++){
    sum+= f[i];
  }
  sum += f[N-1]/2;
  sum *= h;
  return sum;
}

/* Compute integral using Simpson's rule, N must be odd */
double simpint(double f [N]){
  double sum;
  sum= f[0];
  for(i=1; i<N-1; i+=2){
    sum+= 4*f[i];
  }
  for(i=2; i<N-2; i+=2){
    sum+= 2*f[i];
  }
  sum += f[N-1];
  sum *= h/3;
  return sum;
}

/* Updates the derivative of a function using numerical differentiation
   Assumes f is a function of r. */
void updatederiv(double f [N], double (&dfdr) [N]){  
  // have to use forward difference
  dfdr[0]= (f[1]-f[0])/h;
  // use central difference for middle ones
  for (i=1; i< N-1; i++){
    dfdr[i]= 0.5*(f[i+1]-f[i-1])/h;
  }
  // use backward difference
  dfdr[N-1]= (f[N-2]-f[N-1])/h;
}

/* Compute the kinetic energy of a wavefunction */
double calcEk(double u [N]){
  double Ek;
  updatederiv(u, dudr);
  for (i=1; i<N; i++){
    dudrsq[i]= dudr[i]*dudr[i];  
  }
  Ek = hbsq_m*0.5*simpint(dudrsq);
  return Ek;
}

/* Compute the potential energy of a wavefunction using a 
   predetermined potential*/
double calcEp(double u [N]){
  for (i=1; i<N; i++){
    eptmp[i]= (-Z*esq/r[i] + 0.5*V[i])*usq[i];
  }
  return simpint(eptmp);
}

/* Use the current wavefunction to set the right hand side of 
   Poisson's equation. */
void setf(){
  updateusq();
  f[0]= 0.0; // boundary condition
  for (i=1; i<N; i++){
    f[i]= -esq*usq[i]/r[i];
  }
}


/* Uses the Numerov method to update the (k+1)th term of u
   k must be in range (2, N-2). Assumes terms
   u[0] through u[k] are already initialized and
   w is initialized. */
void numerovf(int k, double (&u) [N], double w [N], double s [N]){
  double b = h*h/12.0;
  u[k+1]= ( 2*(1-5*b*w[k])*u[k] - (1+ b*w[k-1])*u[k-1]
	    + b*(s[k+1]+10*s[k]+s[k-1]) )/(1+b*w[k+1]);
}

/* Uses the Numerov method to update the (k-1)th term of u
   k must be in range (2, N-2). Assumes terms
   u[0] through u[k] are already initialized and
   w is initialized. */
void numerovb(int k, double (&u) [N], double w [N], double s [N]){
  double b = h*h/12.0;
  u[k-1]= ( 2*(1-5*b*w[k])*u[k] - (1+ b*w[k+1])*u[k+1]
	    + b*(s[k+1]+10*s[k]+s[k-1]) )/(1+b*w[k-1]);
}

/* Solves Poisson's equation del^2 v = f*/
void calcv(){
  // initial conditions
  v[0]= 0.0; // require v[0] = 0 since v(r)/r = V(r)
  v[1]= 0.0; // v[1] is roughly 0.0
  for (i=1; i<N-1; i++){
    numerovf(i, v, blank, f);
  }
  // Need correction using Konnin's method p.54
  double m = (v[N-1]-v[N-11])/(h*10);
  // cout << "m = " << m << endl;
  double b = v[N-1]-m*(N-1)*h;
  // cout << "b = " << b << endl;
  for (i=0; i<N; i++){
    v[i]= v[i]- m*h*i;
  }
}

/* Calculate V(r) for use in the potential energy calculation. */
void calcV(){
  for(i=1; i<N; i++){
    V[i]= v[i]/r[i];
  }
  // linear approximation for V(r=0)
  V[0]= (V[1]-V[2])+V[1];
}

// write u(r) output to a file
void writeutofile(){
  stringstream fname;
  // cout << t << endl;
  fname << "data/u_" << numiter << ".dat";
  ofstream fp;
  fp.open(fname.str().c_str());
  if ( fp.fail() ){
    cout << "cannot open file" << endl;
  }
  for (i=0; i<N; i++){
    fp << r[i] << setw(15) << u[i] << setw(15) << endl;
    // << w[i] << setw(15)
    // << dudr[i] << setw(15) << R[i] << setw(15)
    // << f[i] << setw(15) << v[i] << 
      
  }
  fp.close();
}

// write R(r) output to file last
void writeRtofile(){
  ofstream fp;
  fp.open("hf1d_he.dat");
  if ( fp.fail() ){
    cout << "cannot open file" << endl;
  }
  for (i=0; i<N; i++){
    fp << r[i] << setw(15) << R[i] << setw(15) << endl;
    // << w[i] << setw(15)
    // << dudr[i] << setw(15) << R[i] << setw(15)
    // << f[i] << setw(15) << v[i] <<       
  }
  fp.close();
}


/* Determine the weighing function for use in calculating
   unew using Numerov's method. */
void calcw(double E){
  for(i=1; i<N; i++){
    w[i]= (2/hbsq_m)*(Z*esq/r[i] - 0.5*V[i] + E);
  }
  w[0]= 2*w[1]-w[2]; // linear extrapolation
}

/* Given a value of the total energy, use Numerov's method
   to trace a new wavefunction and normalize it accordingly. */
void calcunew(){
  unew[N-1]= 0.1*u[N-1];
  unew[N-2]= 0.1*u[N-2]; // initial conditions
  // for (i=1; i<N/10; i++){
  //   numerovf(i, unew, w, blank);
  // }
  for (i=N-2; i>1; i--){
    numerovb(i, unew, w, blank);
  }
  // normalization required
  for (i=0; i<N; i++){
    unewsq[i]= unew[i]*unew[i];
  }
  double m = sqrt(simpint(unewsq));
  // cout << "norm. const. = " << m << endl;
  for (i=0; i<N; i++){
    unew[i]/= m;
    unewsq[i]= unew[i]*unew[i];
  }
  // cout << "normalized to: " << simpint(unewsq) << endl;
}

void updateR(){
  for (i=1; i<N; i++){
    R[i]= u[i]/(sqrt(4*pi)*r[i]);
   }
}

void scfiter(){
  setf(); // del^2 v = f (Poisson's equation)
  calcv();
  calcV();
  Ep= calcEp(u);
  Ek= calcEk(u);
  E= Ep+Ek;
  cout << "kinetic energy is: " << Ek << endl;
  cout << "potential energy is: " << Ep << endl;
  cout << "total energy is: " << E << endl;
  calcw(E); // weighing function to solve for new u
  calcunew();
  // update old values for comparison
  for (i=0; i<N; i++){
    u[i]= unew[i];
  }
  numiter++;
  cout << "iteration " << numiter << " completed." << endl;
  writeutofile();
}

int main(){
  // normalization test
  initialize();
  writeutofile(); // write initial u
  Eold= -50.0; // start E at some value
  E= 0.0;
  while (abs(E-Eold) > 0.001 && numiter < 15){
    Eold= E;
    scfiter();
  }
  Ep= calcEp(u);
  Ek= calcEk(u);
  E= Ep+Ek;
  cout << "final kinetic energy is: " << Ek << endl;
  cout << "final potential energy is: " << Ep << endl;
  cout << "Ek/Ep is: " << Ek/Ep << endl;
  cout << "final total energy is: " << E << endl;
  cout << "stepsize was: " << h << endl;
  cout << "cutoff radius was: " << rf << endl;
  writeRtofile();  
  return(EXIT_SUCCESS);
} // end main()


\end{verbatim}
\end{document}
